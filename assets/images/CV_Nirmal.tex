% LaTeX file for resume.cls 
% This file uses the resume document class (res.cls)

\documentclass[margin]{res} 
% the margin option causes section titles to appear to the left of body text 
\textwidth=5.2in % increase textwidth to get smaller right margin
%\usepackage{helvetica} % uses helvetica postscript font (download helvetica.sty)
%\usepackage{}   % uses new century schoolbook postscript font 
%\usepackage[utf8]{inputenc}
\usepackage{array}
\usepackage{wrapfig}
\usepackage{multirow}
\usepackage{tabu}
\usepackage{amsmath}
\usepackage{hyperref}

\begin{document} 
\name{\makebox[\textwidth][l]{\Large Nirmal Jayaprasad Nair}} % the \\[12pt] adds a blank line after name
 
\address{ \rule{120mm}{2pt}}
\address{ \rule{50mm}{2pt}}
    
\begin{resume}  
\section{\large Contact Information}
\noindent
\makebox[3.5in][l]{University of Illinois at Urbana-Champaign} \\
\makebox[3.5in][l]{Department of Aerospace Engineering} website: nirmaljp6.com\\
\makebox[3.5in][l]{Talbot Laboratory} phone: 217-305-1356           \\
\makebox[3.5in][l]{Urbana, Illinois 61801} email: njn2@illinois.edu

\section{\large Education} 
{\bf University of Illinois at Urbana-Champaign}, Urbana, Illinois \hfill 2018 -- present\\
\hspace{0.5cm} Ph.D., Aerospace Engineering \hfill GPA: 3.96/4.0\\
Concentration: Computational Science and Engineering

{\bf University of Illinois at Urbana-Champaign}, Urbana, Illinois \hfill 2016 -- 2018\\
\hspace{0.5cm} M.S., Aerospace Engineering \hfill GPA: 4.0/4.0

{\bf Indian Institute of Technology Gandhinagar}, Gujarat, India \hfill 2012 -- 2016\\
B.Tech. (\textit{Honors}), Mechanical Engineering \hfill GPA: 9.24/10.0

\section{\large Research Interests}
\begin{table}[h]
\begin{tabular*}{\textwidth}[t]{ll}
\textbullet \  Computational fluid dynamics & \textbullet \  Deep learning \\
\textbullet \  Fluid-structure interaction & \textbullet \  Reduced-order modeling \\
\textbullet \  High performance computing & \textbullet \  Machine learning
\end{tabular*}
\end{table}

\section{\large Research Experience}
{\bf Graduate Research Assistant} \hfill September 2018 -- present\\
Ph.D. thesis adviser: Prof. Andres Goza, UIUC, Champaign, IL \\
%
Integrating sensor data into reduced-order models (ROMs) using deep learning.
\begin{itemize}
\item Developing a state estimation methodology where real-time sensor data is mapped to the ROM state space using deep neural networks in Pytorch.
%\vspace{-0.15cm}
\end{itemize}

Data-driven flow field estimation around an airfoil using passively deployed flaps.
\begin{itemize}
\item Passively deployed flaps will be modeled as flow-field estimating sensors by mapping the dynamics of the flaps to the flow-field using snapshot data.
%\vspace{-0.15cm}
\end{itemize}

Scalable solver for simulating strongly coupled fluid-interaction flows.
\begin{itemize}
\item Developing a parallel CFD solver for simulating fluid-structure interaction problems consisting of rigid and torsionally hinged bodies using MPI and PETSC.
\end{itemize}

{\bf CSRI Summer Intern} \hfill June 2019 -- August 2019\\
Adviser: Dr. Kevin Carlberg, Sandia National Laboratories, Livermore, CA \\
%
Guaranteeing convergence of ROMs on nonlinear manifold using transfer learning.
\begin{itemize}
\item Developed an adaptive manifold refinement strategy to enable convergence of ROMs on manifolds built using deep convolutional autoencoders on Tensorflow.
%\vspace{-0.15cm}
\end{itemize}

{\bf Graduate Research Assistant} \hfill August 2016 -- August 2018\\
M.S. thesis adviser: Prof. Maciej Balajewicz, UIUC, Champaign, IL \\
Data-driven reduced-order modeling of advection-dominated fluid flows.
\begin{itemize}
\item Developed a novel data-driven model order reduction method for parametric, steady-state fluid flows containing evolving shocks.
%\vspace{-0.15cm}
%\item Demonstrated the computational efficiency of the proposed approach on several CFD problems. 
\end{itemize}


{\bf Summer Undergraduate Research Fellow} \hfill May -- July 2015\\
Adviser: Prof. Austin Minnich, California Institute of Technology, Pasadena, CA
\begin{itemize}
\item Designed and fabricated a prototype consisting of thermoelectric generators to
power wireless temperature sensors in aircraft.
\end{itemize}

{\bf Summer Research Internship Program} \hfill   May 2014 -- April 2015\\
Adviser: Prof. Vinod Narayanan, IIT Gandhinagar, India
\begin{itemize}
\item Studied the stability characteristics of axisymmetric thermal boundary layer of various fluids in response to heating and cooling.
\end{itemize}

%\newpage

\section{\large Skills}		 
Programming: Python, Matlab, Fortran, C.\\
Machine Learning: Pytorch, Tensorflow.\\
High Performance Computing: PETSC, MPI, OpenMP.\\
CFD and CAD: Ansys Fluent, Star CCM+, Autodesk Inventor.\\ 
Miscellaneous: Latex, Git.

\section{\large Publications}
 \begin{itemize}
 \item[1.] N.J. Nair and A. Goza. Leveraging reduced-order models for state estimation using deep learning. \textit{Journal of Fluid Mechanics}, 897, 2020.
 
 \item[2.] N.J. Nair and M. Balajewicz. Transported snapshot model order reduction approach for parametric, steady-state fluid flows containing parameter-dependent shocks. \textit{International Journal for Numerical Methods in Engineering}, 2019; \linebreak 117:1234–1262.
 
  \item[3.] N. Jayaprasad, et al. Exploring viscous damping in undergraduate Physics laboratory using electromagnetically coupled oscillators. \textit{Preprint arXiv:1311.7489}, 2013.
 \end{itemize}
 
\section{\large Conference Talks}

\textbf{Invited Talks}

\begin{itemize} 
 \item[1.] N.J. Nair and M. Balajewicz. Transported snapshot model order reduction approach for parametric, steady-state fluid flows containing parameter dependent shocks. \textit{SIAM Conference on Computational Science and Engineering}, 2019.
\end{itemize}

\textbf{Contributed Talks} 
\begin{itemize}
 \item[1.] N.J. Nair and A. Goza. Integrating sensor data into reduced-order models with deep learning. \textit{APS, Division of Fluid Dynamics}, 2019.

 \item[2.] N.J. Nair and M. Balajewicz. Physics based interpolation for steady parametric partial differential equations. \textit{APS, Division of Fluid Dynamics}, 2017.
 
 \item[3.] N.J. Nair and U. Shah. A simple computational tool for studying acoustic waves in nonlinear medium. \textit{ASME, International Design Engineering Technical Conferences}, 2017.
 
 \item[4.] N. Jayaprasad and V. Narayanan. Effect of viscosity stratification on stability of axisymmetric boundary layer. \textit{APS, Division of Fluid Dynamics}, 2015.
  
\end{itemize}
		 
%\section{\large Course Projects}
%{\bf Nonlinear lifting line theory on biplanes for low speed applications}\\
%Instructor: Philip Ansell  \hfill February - April 2017
%\begin{itemize}
%\item Used nonlinear lifting line theory to study the load distribution on a biplanar wing for different biplane configurations specifically for flying car applications.
%\end{itemize}
%
%{\bf Shock induced flow separation control over Onera M6 wing using vortex generators} \hfill September - November 2016\\
%Instructor: Philip Ansell 
%\begin{itemize}
%\item Studied passive flow control using vortex generators to delay shock induced flow separation on Onera M6 wing in transonic flow regime using Ansys Fluent.
%\end{itemize}
%
%{\bf Modeling blood flow in renal artery} \hfill October - November 2015\\
%Instructor: Murali Damodaran 
%\begin{itemize}
%\item Simulated non-Newtonian blood flow in renal artery using Star CCM+.
%\end{itemize}
		 
\section{\large Honors \& Awards}
AE Outstanding Graduate Student Fellowship, UIUC \hfill 2020\vspace{0.1cm}\\
Conference Travel Grant, APS DFD \hfill 2019\vspace{0.1cm}\\
SIAM Student Travel Award, SIAM CSE \hfill 2019\vspace{0.1cm}\\
Conference Travel Award for Graduate Students, UIUC \hfill 2017\vspace{0.1cm}\\
MSNDC Student Travel Grant, ASME IDETC \hfill 2017\vspace{0.1cm}\\
Award for `Best Performance in the core subjects of Engineering Graphics, Manufacturing and Workshop Practice', IIT Gandhinagar \hfill 2016\vspace{0.1cm}\\
Summer Undergraduate Research Fellowship, Caltech \hfill 2015\vspace{0.1cm}\\
Dean's List, IIT Gandhinagar \hfill 2013, 2014, 2015\vspace{0.1cm}\\
Merit cum Means Scholarship, IIT Gandhinagar \hfill 2012, 2013, 2014\vspace{0.1cm}\\
Winner of Ricoh Printer Design Challenge, IIT Gandhinagar \hfill 2014

%\section{\large Relevant Coursework} 
%Applied Aerodynamics; Wing Theory; Unsteady Aerodynamics; Viscous Flow; Computational Fluid Dynamics; Uncertainty Quantification; Aeroelasticity; Mathematical Methods.

\section{\large Projects}

{\bf Nonlinear modal decomposition of transient fluid flows}, UIUC \hfill Spring 2020
\begin{itemize}
\item Developed a nonlinear modal decomposition framework to accurately represent transient fluid flows and extract meaningful flow structures using deep convolution autoencoders.
\end{itemize}

{\bf MPI and OpenMP based parallel 2D CFD solver}, UIUC \hfill Fall 2018
\begin{itemize}
\item Developed a parallel solver based on finite difference methods to solve the 2D advection-diffusion equation using MPI and OpenMP.
\end{itemize}

{\bf Passive flow control using vortex generators}, UIUC \hfill Fall 2016
\begin{itemize}
\item Studied passive flow control using vortex generators to delay shock induced flow separation on an Onera M6 wing in transonic flow regime using Ansys Fluent.
\end{itemize}

\section{\large Academic Services and Affiliations}		 
Reviewer, Journal of Computational Physics \hfill 2018--\vspace{0.1cm}\\
Reviewer, International Journal for Numerical Methods in Engineering \hfill 2018--\vspace{0.1cm}\\
Student Member, Society of Industrial and Applied Mathematicians (SIAM) \hfill 2018 --\vspace{0.1cm}\\
Student Member, American Physical Society (APS) \hfill 2017 --
		 
\section{\large Teaching}
{\bf Teaching Assistant}, UIUC \hfill September -- December 2019 \\
AE 433: Aerospace Propulsion
%Led revision sessions for sophomore students to clarify doubts and revise difficult concepts under the Peer Assisted Learning (PAL) program.
\vspace{-0.1cm}

{\bf Tutor}, IIT Gandhinagar \hfill January -- March 2016 \\
ES 212: Momentum, Heat and Mass Transfer
%Led revision sessions for sophomore students to clarify doubts and revise difficult concepts under the Peer Assisted Learning (PAL) program.
\vspace{-0.1cm}

{\bf Teaching Assistant}, IIT Gandhinagar \hfill August -- November 2013 \\
ES 101: Engineering Graphics
%Designed and led lab sessions on using Autodesk Inventor and graded the engineering drawing lab assignments for freshman students.

%\section{\large Volunteering Experience}
%\begin{itemize}
%\item[1.] Reviewer: Reviewed manuscripts for the AIAA Journal and the Journal of Computational Physics.
%\item[2.] Volunteer: Volunteered in setting up and representing the booth of the Aerospace Engineering Department, UIUC at Airventure, Oshkosh 2018.
%\end{itemize}

\section{\large Leadership}
{\bf Mentor}, Summer Undergraduate Research, UIUC \hfill May -- July 2017 \\
Mentored and supervised an undergraduate student on his research project and provided the necessary guidance to maintain progress.

{\bf Events Coordinator}, Amalthea' 13, IIT Gandhinagar \hfill May -- October 2013 \\
Led a team of 21 students to plan and organize various technical events at Amalthea'13 which is the annual technical summit of IIT Gandhinagar.

%\section{\large References}
%
%\noindent
%\makebox[3.5in][l]{Dr. Andres Goza} \\
%\makebox[3.5in][l]{Assistant Professor, Department of Aerospace Engineering} \\
%\makebox[3.5in][l]{University of Illinois at Urbana-Champaign}        \\
%\makebox[3.5in][l]{email: agoza@illinois.edu}        
%
%\noindent
%\makebox[3.5in][l]{Dr. Maciej Balajewicz} \\
%\makebox[3.5in][l]{Assistant Professor, Department of Aerospace Engineering} \\
%\makebox[3.5in][l]{University of Illinois at Urbana-Champaign}         \\  
%\makebox[3.5in][l]{email: mbalajew@illinois.edu}           

\end{resume}
\end{document} 



