% LaTeX file for resume.cls 
% This file uses the resume document class (res.cls)

\documentclass[margin]{res} 
% the margin option causes section titles to appear to the left of body text 
\textwidth=5.2in % increase textwidth to get smaller right margin
%\usepackage{helvetica} % uses helvetica postscript font (download helvetica.sty)
%\usepackage{}   % uses new century schoolbook postscript font 
%\usepackage[utf8]{inputenc}
\usepackage{array}
\usepackage{wrapfig}
\usepackage{multirow}
\usepackage{tabu}
\usepackage{amsmath}
\usepackage{hyperref}

\begin{document} 
\name{\makebox[\textwidth][l]{\Large Nirmal J. Nair \hfill \normalsize \normalfont email: nirmaljp6@gmail.com $|$ tel: 217-305-1356 $|$ website: nirmaljp6.com}} % the \\[12pt] adds a blank line after name
 
\address{ \rule{120mm}{2pt}}
\address{ \rule{50mm}{2pt}}
    
\begin{resume}  
%\section{\large Contact Information}
%\noindent
%\makebox[3.5in][l]{University of Illinois at Urbana-Champaign} \\
%\makebox[3.5in][l]{Department of Aerospace Engineering} website: nirmaljp6.com\\
%\makebox[3.5in][l]{Talbot Laboratory} phone: 217-305-1356           \\
%\makebox[3.5in][l]{Urbana, Illinois 61801} email: njn2@illinois.edu

\section{\large Education} 
{\bf University of Illinois at Urbana-Champaign}, Urbana, IL \hfill 2018--Dec 2022 (exp)\\
\hspace{0.5cm} Ph.D., Aerospace Engineering \hfill GPA: 3.96/4.0\\
Concentration: Computational Science and Engineering

{\bf University of Illinois at Urbana-Champaign}, Urbana, IL \hfill 2016--2018\\
\hspace{0.5cm} M.S., Aerospace Engineering \hfill GPA: 4.0/4.0

{\bf Indian Institute of Technology Gandhinagar}, Gujarat, India \hfill 2012--2016\\
B.Tech. (\textit{Honors}), Mechanical Engineering \hfill GPA: 9.24/10.0

\section{\large Interests}
\begin{table}[h]
\begin{tabular*}{\textwidth}[t]{ll}
\textbullet \  Computational fluid dynamics & \textbullet \  Deep learning \\
\textbullet \  Numerical modeling  & \textbullet \  Reinforcement learning \\
\textbullet \  High performance computing & \textbullet \  Reduced-order modeling
\end{tabular*}
\end{table}

\section{\large Experience}
{\bf University of Illinois at Urbana-Champaign} \hfill Urbana, IL \\
{\it Graduate Research Assistant} \hfill Sept 2018--present\\
Ph.D. thesis adviser: Prof. Andres Goza \\
%
{\textbf{Project:} Computational fluid dynamics (CFD) of an aeroelastic flap for flow control}
\begin{itemize}
	\item Computationally modeled the multi-physics flow over an airfoil mounted with a bio-inspired aeroelastic flap for improving aerodynamic performance.
	\item Performed a multi-dimensional parametric study, k-means clustering and detailed data analysis to identify features that enhanced aerodynamic lift up to 27\%.
	\item Collaborated with a research group at Princeton University to correlate CFD simulations with wind-tunnel experiments.
\end{itemize}

{\textbf{Project:}  Machine learning for state estimation and active flow control}
\begin{itemize}
	\item Developed a nonlinear state estimation approach where real-time sensor data was used to accurately predict the full-flow state using deep neural networks.
	\item Currently using deep reinforcement learning for designing a closed-loop controller for active flow control of an airfoil mounted with a controllable aeroelastic flap.
	\item Utilized methodologies such as proximal policy optimization, principal component analysis and multi-layer perceptrons on PyTorch.
\end{itemize}

{\textbf{Project:}  High performance computing and development of efficient algorithms}
\begin{itemize}
	\item Developed a scalable multi-physics (fluid-structure interaction) solver involving the Navier-Stokes and Newton's equations using PETSC and MPI. 
	\item Designed an efficient algorithm that addressed a critical computational bottleneck related to the computation of a fluid-structure coupling matrix.
	\item Achieved 4--10x increase in serial computational speed and $\sim$84\% parallel strong scaling efficiency.
\end{itemize}

{\bf Sandia National Laboratories} \hfill Livermore, CA \\
{\it Computer Science Research Institute Summer Intern} \hfill June--August 2019\\
Adviser: Dr. Kevin Carlberg \\
%
\textbf{Project:} Transfer learning for enabling convergence of reduced-order models (ROM)
\begin{itemize}
\item Formulated an adaptive refinement strategy to guarantee convergence of nonlinear ROMs built using deep convolutional autoencoders on TensorFlow.
\item Developed methodologies for efficiently retraining selected weights of the autoencoder and real-time augmentation of the latent space.
%\vspace{-0.15cm}
\end{itemize}

{\bf University of Illinois at Urbana-Champaign} \hfill Urbana, IL \\
{\it Graduate Research Assistant} \hfill Aug 2016--Aug 2018\\
M.S. thesis adviser: Dr. Maciej Balajewicz\\
\textbf{Project:} Data-driven reduced-order modeling of fluid flows
\begin{itemize}
	\item Identified a critical drawback of linear ROMs in accurately predicting fluid flows that are dominated by advection and strong discontinuities.
	\item Developed a hyper-reduced, physics-based ROM where solutions are obtained by a nonlinear transformation of a linear subspace using collected flow data.
	\item Achieved 300-10000x increase in computational speed while incurring only $\sim$1\% error when tested on several CFD problems. 
\end{itemize}

{\bf California Institute of Technology} \hfill Pasadena, CA\\
{\it Summer Undergraduate Research Fellow} \hfill May--July 2015\\
%Adviser: Prof. Austin Minnich\\
\textbf{Project:} Thermoelectric generators for waste heat scavenging in aircraft
\begin{itemize}
	\item Designed and fabricated a thermal-electrical prototype consisting of a thermoelectric generator, heat fin and electronics to generate electricity from waste heat. 
	\item Powered a wireless temperature sensor from a temperature difference of $5^\circ$ K.
\end{itemize}

{\bf IIT Gandhinagar} \hfill India\\
{\it Summer Research Intern} \hfill   May 2014--April 2015\\
%Adviser: Prof. Vinod Narayanan\\
\textbf{Project:} Stability analysis of thermal boundary layers
\begin{itemize}
	\item Performed local stability analysis to quantify the effect of heating on viscosity and the stability characteristics of axisymmetric thermal boundary layers.%in response to heating and cooling.
\end{itemize}

%\newpage

\section{\large Publications}
 \begin{itemize}
 
  	\item[1.] \textbf{N.J. Nair} and A. Goza. Fluid-structure interaction of a bio-inspired passively deployable flap for lift enhancement. \textit{Preprint arXiv:2203.00037}, 2022. (Under review at \textit{Physical Review Fluids}).

	\item[2.] \textbf{N.J. Nair} and A. Goza. A strongly coupled immersed boundary method for fluid-structure interaction that mimics the efficiency of stationary body methods. \textit{Journal of Computational Physics},  110897, 2022.
	
	\item[3.] \textbf{N.J. Nair}, Z. Flynn and A. Goza. Numerical study of multiple bio-inspired torsionally hinged flaps for passive flow control. \textit{Fluids}, 7(2), 44, 2022.
	
	\item[4.] \textbf{N.J. Nair} and A. Goza. Leveraging reduced-order models for state estimation using deep learning. \textit{Journal of Fluid Mechanics}, 897, 2020.
	
	\item[5.] \textbf{N.J. Nair} and M. Balajewicz. Transported snapshot model order reduction approach for parametric, steady-state fluid flows containing parameter-dependent shocks. \textit{International Journal for Numerical Methods in Engineering}, 2019; \linebreak 117:1234–1262.
	
 \end{itemize}
 
\section{\large Conference Talks and Proceedings}

\textbf{Invited}

\begin{itemize} 
	
 \item[1.] \textbf{N.J. Nair} and A. Goza. Active flow control of a covert-inspired deployable flap strategy using reinforcement learning. \textit{USNC, TAM}, 2022. (Accepted).
 
 \item[2.] A. Goza and \textbf{N.J. Nair}. Effects of flap-vortex interactions on the lift of an airfoil mounted with a passively deployable flap. \textit{DisCoVor, EPFL}, 2022. (Accepted).
 
 \item[3.] \textbf{N.J. Nair} and M. Balajewicz. Transported snapshot model order reduction approach for parametric, steady-state fluid flows containing parameter dependent shocks. \textit{SIAM CSE}, 2019.
 
\end{itemize}


\newpage
\textbf{Contributed} 
\begin{itemize}

 \item[1.] A.K. Othman, \textbf{N.J. Nair}, A. Sandeep, A. Goza and A.Wissa. Numerical and experimental study of a covert-inspired
passively deployable flap for aerodynamic lift enhancement. \textit{AIAA Aviation}, 2022. (Accepted)

 \item[2.] \textbf{N.J. Nair} and A. Goza. Effects of Torsional Stiffness and Inertia on a Passively Deployable Flap for Aerodynamic Lift Enhancement. \textit{AIAA Scitech Forum}, 2022.
 
 \item[3.] \textbf{N.J. Nair} and A. Goza. Approaching the efficiency of stationary-body methods in a strongly coupled immersed boundary framework for fluid-structure interaction. \textit{APS, Division of Fluid Dynamics}, 2021.
	
 \item[4.] \textbf{N.J. Nair} and A. Goza. Numerical study of a passively deployable flap for aerodynamic flow control. \textit{APS, Division of Fluid Dynamics}, 2020.
 
 \item[5.] \textbf{N.J. Nair} and A. Goza. Integrating sensor data into reduced-order models with deep learning. \textit{APS, Division of Fluid Dynamics}, 2019.

 \item[6.] \textbf{N.J. Nair} and M. Balajewicz. Physics based interpolation for steady parametric partial differential equations. \textit{APS, Division of Fluid Dynamics}, 2017.
 
 \item[7.] \textbf{N.J. Nair} and U. Shah. A simple computational tool for studying acoustic waves in nonlinear medium. \textit{ASME, IDETC}, 2017.
 
 \item[8.] \textbf{N.J. Nair} and V. Narayanan. Effect of viscosity stratification on stability of axisymmetric boundary layer. \textit{APS, Division of Fluid Dynamics}, 2015.
  
\end{itemize}
		 
%\section{\large Course Projects}
%{\bf Nonlinear lifting line theory on biplanes for low speed applications}\\
%Instructor: Philip Ansell  \hfill February - April 2017
%\begin{itemize}
%\item Used nonlinear lifting line theory to study the load distribution on a biplanar wing for different biplane configurations specifically for flying car applications.
%\end{itemize}
%
%{\bf Shock induced flow separation control over Onera M6 wing using vortex generators} \hfill September - November 2016\\
%Instructor: Philip Ansell 
%\begin{itemize}
%\item Studied passive flow control using vortex generators to delay shock induced flow separation on Onera M6 wing in transonic flow regime using Ansys Fluent.
%\end{itemize}
%
%{\bf Modeling blood flow in renal artery} \hfill October - November 2015\\
%Instructor: Murali Damodaran 
%\begin{itemize}
%\item Simulated non-Newtonian blood flow in renal artery using Star CCM+.
%\end{itemize}
		 
\section{\large Honors \& Awards}
Kuck Computational Science and Engineering Scholarship, UIUC \hfill 2022\vspace{0.1cm}\\
AE Outstanding Graduate Student Fellowship, UIUC \hfill 2020\vspace{0.1cm}\\
Conference Grant, APS DFD \hfill 2019\vspace{0.1cm}\\
SIAM Conference Student Award, SIAM CSE \hfill 2019\vspace{0.1cm}\\
Conference Award for Graduate Students, UIUC \hfill 2017\vspace{0.1cm}\\
MSNDC Conference Student Grant, ASME IDETC \hfill 2017\vspace{0.1cm}\\
Commencement award for `Best Performance in the core subjects of Engineering Graphics, Manufacturing and Workshop Practice', IIT Gandhinagar \hfill 2016\vspace{0.1cm}\\
Summer Undergraduate Research Fellowship, Caltech \hfill 2015\vspace{0.1cm}\\
Dean's List, IIT Gandhinagar \hfill 2013, 2014, 2015\vspace{0.1cm}\\
Merit cum Means Scholarship, IIT Gandhinagar \hfill 2012, 2013, 2014\vspace{0.1cm}\\
Winner of Ricoh Printer Design Challenge, IIT Gandhinagar \hfill 2014

%\section{\large Relevant Coursework} 
%Applied Aerodynamics; Wing Theory; Unsteady Aerodynamics; Viscous Flow; Computational Fluid Dynamics; Uncertainty Quantification; Aeroelasticity; Mathematical Methods.

\section{\large Selected Projects}

{\bf Aeroacoustics of vortex shedding around a stalled airfoil}, UIUC \hfill Spring 2021
\begin{itemize}
\item Predicted noise due to vortex shedding around an airfoil by numerically solving the Ffowcs Williams-Hawkings equation using the Farassat Formulation 1A.
\end{itemize}

{\bf Nonlinear modal decomposition of transient fluid flows}, UIUC \hfill Spring 2020
\begin{itemize}
\item Developed a nonlinear modal decomposition framework to identify meaningful flow structures in transient fluid flows using deep convolutional autoencoders.
\end{itemize}

{\bf Commercial software for computational fluid dynamics}
\begin{itemize}
	\item Performed CFD simulations on Ansys Fluent to study passive flow control using vortex generators on an Onera M6 wing at UIUC in Fall 2016.
	\item Simulated non-Newtonian blood flow in an artery using Star CCM+ to
	study the effect of blockages on blood pressure at IIT Gandhinagar in Spring 2016.
\end{itemize}

%{\bf Passive flow control using vortex generators}, UIUC \hfill Fall 2016
%\begin{itemize}
%\item Performed CFD simulations on Ansys Fluent to study passive flow control using vortex generators to delay shock induced flow separation on an Onera M6 wing.
%\end{itemize}
		 
%\section{\large Teaching}
%{\bf Teaching Assistant}, UIUC \hfill Fall 2019, Fall 2020 \\
%AE 433: Aerospace Propulsion
%Led revision sessions for sophomore students to clarify doubts and revise difficult concepts under the Peer Assisted Learning (PAL) program.
%\vspace{-0.1cm}

%{\bf Tutor}, IIT Gandhinagar \hfill Spring 2016 \\
%ES 212: Momentum, Heat and Mass Transfer
%Led revision sessions for sophomore students to clarify doubts and revise difficult concepts under the Peer Assisted Learning (PAL) program.
%\vspace{-0.1cm}

%{\bf Teaching Assistant}, IIT Gandhinagar \hfill Fall 2013 \\
%ES 101: Engineering Graphics
%Designed and led lab sessions on using Autodesk Inventor and graded the engineering drawing lab assignments for freshman students.

%\section{\large Volunteering Experience}
%\begin{itemize}
%\item[1.] Reviewer: Reviewed manuscripts for the AIAA Journal and the Journal of Computational Physics.
%\item[2.] Volunteer: Volunteered in setting up and representing the booth of the Aerospace Engineering Department, UIUC at Airventure, Oshkosh 2018.
%\end{itemize}

\section{\large Skills}		 
Programming: Python, MATLAB, Fortran, C.\\
Machine Learning: PyTorch, TensorFlow, Stable Baselines.\\
High Performance Computing: PETSC, MPI, OpenMP.\\
CFD and CAD: Ansys Fluent, Star CCM+, Autodesk Inventor.\\ 
Miscellaneous: Git (Version Control), Simulink, Latex.

\section{\large Leadership}
{\bf University of Illinois at Urbana-Champaign} \hfill Urbana, IL\\
{\it Coordinator}, Upward Bound \hfill June--July 2021
\begin{itemize}
	\item Designed and co-organized a two-day glider building workshop for high school students from underrepresented and minority groups.
	\item Facilitated the procurement of supplies required for the activity and led one of the online Zoom sessions to guide the students through the workshop.
\end{itemize}

{\it Teaching Assistant} \hfill Fall 2019, Fall 2020
\begin{itemize}
	\item Led regular office hours, exam revision sessions and created tutorial problems to aid the students in the AE433: Aerospace Propulsion course.
	\item Listed in the ``List of Teachers Ranked as Excellent for Fall 2019'' at UIUC.
\end{itemize}


{\it Mentor}
\begin{itemize}
	\item As a senior graduate student, mentored newly admitted graduate students on the software architecture of our group.
	\item Mentored and supervised an undergraduate student on their research project on numerical modeling and CFD in summer 2017.
\end{itemize}

{\bf IIT Gandhinagar} \hfill India\\
{\it Events Coordinator}, Amalthea' 13 \hfill May--Oct 2013
\begin{itemize}
	\item Led a team of 21 students to plan and organize various technical events at Amalthea'13, the annual technical summit of IIT Gandhinagar.
	%\item Managed 7 sub-teams that organized various competitions encompassing robotics, chemistry and quiz. 
\end{itemize}

{\it College Soccer Player} \hfill 2013--2015
\begin{itemize}
	\item Member of the IIT Gandhinagar soccer team and participated at two annual Inter-IIT tournaments.
	\item Captain of one of the five teams in an intra-college soccer league.
\end{itemize}

{\it Undergraduate Lead Teaching Assistant} \hfill Fall 2013
\begin{itemize}
	\item Designed and led the lab sessions on using Autodesk Inventor in the Engineering Graphics course for freshman students.
\end{itemize}

%\section{\large References}
%
%\noindent
%\makebox[3.5in][l]{Dr. Andres Goza} \\
%\makebox[3.5in][l]{Assistant Professor, Department of Aerospace Engineering} \\
%\makebox[3.5in][l]{University of Illinois at Urbana-Champaign}        \\
%\makebox[3.5in][l]{email: agoza@illinois.edu}        
%
%\noindent
%\makebox[3.5in][l]{Dr. Maciej Balajewicz} \\
%\makebox[3.5in][l]{Assistant Professor, Department of Aerospace Engineering} \\
%\makebox[3.5in][l]{University of Illinois at Urbana-Champaign}         \\  
%\makebox[3.5in][l]{email: mbalajew@illinois.edu}           

\end{resume}
\end{document} 



